% Options for packages loaded elsewhere
\PassOptionsToPackage{unicode}{hyperref}
\PassOptionsToPackage{hyphens}{url}
%
\documentclass[
  ignorenonframetext,
  aspectratio=169]{beamer}
\usepackage{pgfpages}
\setbeamertemplate{caption}[numbered]
\setbeamertemplate{caption label separator}{: }
\setbeamercolor{caption name}{fg=normal text.fg}
\beamertemplatenavigationsymbolsempty
% Prevent slide breaks in the middle of a paragraph
\widowpenalties 1 10000
\raggedbottom
\setbeamertemplate{part page}{
  \centering
  \begin{beamercolorbox}[sep=16pt,center]{part title}
    \usebeamerfont{part title}\insertpart\par
  \end{beamercolorbox}
}
\setbeamertemplate{section page}{
  \centering
  \begin{beamercolorbox}[sep=12pt,center]{part title}
    \usebeamerfont{section title}\insertsection\par
  \end{beamercolorbox}
}
\setbeamertemplate{subsection page}{
  \centering
  \begin{beamercolorbox}[sep=8pt,center]{part title}
    \usebeamerfont{subsection title}\insertsubsection\par
  \end{beamercolorbox}
}
\AtBeginPart{
  \frame{\partpage}
}
\AtBeginSection{
  \ifbibliography
  \else
    \frame{\sectionpage}
  \fi
}
\AtBeginSubsection{
  \frame{\subsectionpage}
}

\usepackage{amsmath,amssymb}
\usepackage{iftex}
\ifPDFTeX
  \usepackage[T1]{fontenc}
  \usepackage[utf8]{inputenc}
  \usepackage{textcomp} % provide euro and other symbols
\else % if luatex or xetex
  \usepackage{unicode-math}
  \defaultfontfeatures{Scale=MatchLowercase}
  \defaultfontfeatures[\rmfamily]{Ligatures=TeX,Scale=1}
\fi
\usepackage{lmodern}
\ifPDFTeX\else  
    % xetex/luatex font selection
\fi
% Use upquote if available, for straight quotes in verbatim environments
\IfFileExists{upquote.sty}{\usepackage{upquote}}{}
\IfFileExists{microtype.sty}{% use microtype if available
  \usepackage[]{microtype}
  \UseMicrotypeSet[protrusion]{basicmath} % disable protrusion for tt fonts
}{}
\makeatletter
\@ifundefined{KOMAClassName}{% if non-KOMA class
  \IfFileExists{parskip.sty}{%
    \usepackage{parskip}
  }{% else
    \setlength{\parindent}{0pt}
    \setlength{\parskip}{6pt plus 2pt minus 1pt}}
}{% if KOMA class
  \KOMAoptions{parskip=half}}
\makeatother
\usepackage{xcolor}
\newif\ifbibliography
\setlength{\emergencystretch}{3em} % prevent overfull lines
\setcounter{secnumdepth}{-\maxdimen} % remove section numbering


\providecommand{\tightlist}{%
  \setlength{\itemsep}{0pt}\setlength{\parskip}{0pt}}\usepackage{longtable,booktabs,array}
\usepackage{calc} % for calculating minipage widths
\usepackage{caption}
% Make caption package work with longtable
\makeatletter
\def\fnum@table{\tablename~\thetable}
\makeatother
\usepackage{graphicx}
\makeatletter
\def\maxwidth{\ifdim\Gin@nat@width>\linewidth\linewidth\else\Gin@nat@width\fi}
\def\maxheight{\ifdim\Gin@nat@height>\textheight\textheight\else\Gin@nat@height\fi}
\makeatother
% Scale images if necessary, so that they will not overflow the page
% margins by default, and it is still possible to overwrite the defaults
% using explicit options in \includegraphics[width, height, ...]{}
\setkeys{Gin}{width=\maxwidth,height=\maxheight,keepaspectratio}
% Set default figure placement to htbp
\makeatletter
\def\fps@figure{htbp}
\makeatother

\usepackage{mathpazo}
\usepackage{unicode-math}
\makeatletter
\makeatother
\makeatletter
\makeatother
\makeatletter
\@ifpackageloaded{caption}{}{\usepackage{caption}}
\AtBeginDocument{%
\ifdefined\contentsname
  \renewcommand*\contentsname{Table of contents}
\else
  \newcommand\contentsname{Table of contents}
\fi
\ifdefined\listfigurename
  \renewcommand*\listfigurename{List of Figures}
\else
  \newcommand\listfigurename{List of Figures}
\fi
\ifdefined\listtablename
  \renewcommand*\listtablename{List of Tables}
\else
  \newcommand\listtablename{List of Tables}
\fi
\ifdefined\figurename
  \renewcommand*\figurename{Figure}
\else
  \newcommand\figurename{Figure}
\fi
\ifdefined\tablename
  \renewcommand*\tablename{Table}
\else
  \newcommand\tablename{Table}
\fi
}
\@ifpackageloaded{float}{}{\usepackage{float}}
\floatstyle{ruled}
\@ifundefined{c@chapter}{\newfloat{codelisting}{h}{lop}}{\newfloat{codelisting}{h}{lop}[chapter]}
\floatname{codelisting}{Listing}
\newcommand*\listoflistings{\listof{codelisting}{List of Listings}}
\makeatother
\makeatletter
\@ifpackageloaded{caption}{}{\usepackage{caption}}
\@ifpackageloaded{subcaption}{}{\usepackage{subcaption}}
\makeatother
\makeatletter
\@ifpackageloaded{tcolorbox}{}{\usepackage[skins,breakable]{tcolorbox}}
\makeatother
\makeatletter
\@ifundefined{shadecolor}{\definecolor{shadecolor}{rgb}{.97, .97, .97}}
\makeatother
\makeatletter
\makeatother
\makeatletter
\makeatother
\ifLuaTeX
  \usepackage{selnolig}  % disable illegal ligatures
\fi
\IfFileExists{bookmark.sty}{\usepackage{bookmark}}{\usepackage{hyperref}}
\IfFileExists{xurl.sty}{\usepackage{xurl}}{} % add URL line breaks if available
\urlstyle{same} % disable monospaced font for URLs
\hypersetup{
  pdftitle={Correspondence analysis},
  hidelinks,
  pdfcreator={LaTeX via pandoc}}

\title{Correspondence analysis}
\author{}
\date{}

\begin{document}
\frame{\titlepage}
\ifdefined\Shaded\renewenvironment{Shaded}{\begin{tcolorbox}[breakable, borderline west={3pt}{0pt}{shadecolor}, frame hidden, interior hidden, enhanced, sharp corners, boxrule=0pt]}{\end{tcolorbox}}\fi

\begin{frame}{Introduction to correspondence analysis (CA)}
\protect\hypertarget{introduction-to-correspondence-analysis-ca}{}
\vspace{0.5cm}

\begin{itemize}
\tightlist
\item
  Context: \(\symbf{Y}\) (abundance)

  \begin{itemize}
  \tightlist
  \item
    Goal: Graphically display the relationships between and/or within
    the rows and columns
  \end{itemize}
\end{itemize}

\begin{columns}[T]
\begin{column}{0.45\textwidth}
\begin{center} 

We go from this dataset:

\end{center}

\vspace{0.4cm}

\begin{longtable}[]{@{}lrrrr@{}}
\toprule\noalign{}
& none & light & medium & heavy \\
\midrule\noalign{}
\endhead
SM & 4 & 2 & 3 & 2 \\
JM & 4 & 3 & 7 & 4 \\
SE & 25 & 10 & 12 & 4 \\
JE & 18 & 24 & 33 & 13 \\
SC & 10 & 6 & 7 & 2 \\
\bottomrule\noalign{}
\end{longtable}
\end{column}

\begin{column}{0.025\textwidth}
\end{column}

\begin{column}{0.525\textwidth}
\begin{center}
to this plot:
\end{center}

\vspace{-1cm}

\includegraphics{_tmp/p-cover.pdf}
\end{column}
\end{columns}

\begin{itemize}
\tightlist
\item
  So, the key thing so far is that the rows are not displayed in terms
  of how similar they are overall

  \begin{itemize}
  \tightlist
  \item
    Rather, they're displayed in terms of the similarity of their
    profiles
  \end{itemize}
\end{itemize}
\end{frame}

\begin{frame}{Introduction to correspondence analysis (CA)}
\protect\hypertarget{introduction-to-correspondence-analysis-ca-1}{}
\vspace{0.5cm}

\begin{itemize}
\tightlist
\item
  Context: \(\symbf{Y}\) (abundance)

  \begin{itemize}
  \tightlist
  \item
    Goal: Graphically display the relationships between and/or within
    the rows and columns
  \end{itemize}
\end{itemize}

\begin{columns}[T]
\begin{column}{0.45\textwidth}
\begin{center} 

We go from this dataset:

\end{center}

\vspace{0.4cm}

\begin{longtable}[]{@{}lrrrr@{}}
\toprule\noalign{}
& none & light & medium & heavy \\
\midrule\noalign{}
\endhead
SM & 0.36 & 0.18 & 0.27 & 0.18 \\
JM & 0.22 & 0.17 & 0.39 & 0.22 \\
SE & 0.49 & 0.20 & 0.24 & 0.08 \\
JE & 0.20 & 0.27 & 0.38 & 0.15 \\
SC & 0.40 & 0.24 & 0.28 & 0.08 \\
\bottomrule\noalign{}
\end{longtable}
\end{column}

\begin{column}{0.025\textwidth}
\end{column}

\begin{column}{0.525\textwidth}
\begin{center}
to this plot:
\end{center}

\vspace{-1cm}

\includegraphics{_tmp/p-cover.pdf}
\end{column}
\end{columns}

\begin{itemize}
\tightlist
\item
  So, the key thing so far is that the rows are not displayed in terms
  of how similar they are overall

  \begin{itemize}
  \tightlist
  \item
    Rather, they're displayed in terms of the similarity of their
    profiles
  \end{itemize}
\end{itemize}
\end{frame}

\begin{frame}{Example applications}
\protect\hypertarget{example-applications}{}
\textbf{Datasets}

\begin{itemize}
\tightlist
\item
  Rows are various dams, and columns are counts of waterbird species
\item
  Rows are various immune compartments (e.g.~blood, spleen, lymph), and
  columns are frequencies of immune cell types (e.g.~T cells, B cells,
  NK cells)
\item
  Rows are company brands (e.g.~Cadbury, Beacon, Lindt), and columns are
  consumer ratings on a 1-5 scale (e.g.~quality, price, taste)
\end{itemize}

\textbf{Key characteristics}

\begin{itemize}
\tightlist
\item
  Non-negative
\item
  Natural zero (i.e.~zero means literally nothing and not simply that
  two quantities are equal, for example)
\item
  Same units (e.g.~counts all in thousands)
\end{itemize}

The key property of the data is that proportions make sense throughout.
\end{frame}

\begin{frame}{Correspondence matrix, \(\symbf{P}\)}
\protect\hypertarget{correspondence-matrix-symbfp}{}
\begin{itemize}
\tightlist
\item
  Suppose that we have some matrix \(\symbf{X}:I \times J\) where each
  element

  \begin{itemize}
  \tightlist
  \item
    Rows can be thought of as observations and columns as variables
  \end{itemize}
\item
  The correspondence matrix \(\symbf{P}:I \times J\) is the matrix of
  overall proportions where
\end{itemize}

\[
P_{ij}= \frac{x_{ij}}{\sum_{i=1}^I \sum_{j=1}^J x_{ij}} = \frac{x_{ij}}{n}
\]

\begin{columns}[T]
\begin{column}{0.45\textwidth}
\begin{center} 

We go from $\symbf{X}$

\end{center}

\vspace{0.05cm}

\begin{longtable}[]{@{}lrrrr@{}}
\toprule\noalign{}
& none & light & medium & heavy \\
\midrule\noalign{}
\endhead
SM & 4 & 2 & 3 & 2 \\
JM & 4 & 3 & 7 & 4 \\
SE & 25 & 10 & 12 & 4 \\
JE & 18 & 24 & 33 & 13 \\
SC & 10 & 6 & 7 & 2 \\
\bottomrule\noalign{}
\end{longtable}
\end{column}

\begin{column}{0.025\textwidth}
\end{column}

\begin{column}{0.525\textwidth}
\begin{center}

to $\symbf{P}$

\end{center}

\vspace{0.05cm}

\begin{longtable}[]{@{}lrrrr@{}}
\toprule\noalign{}
& none & light & medium & heavy \\
\midrule\noalign{}
\endhead
SM & 0.02 & 0.01 & 0.02 & 0.01 \\
JM & 0.02 & 0.02 & 0.04 & 0.02 \\
SE & 0.13 & 0.05 & 0.06 & 0.02 \\
JE & 0.09 & 0.12 & 0.17 & 0.07 \\
SC & 0.05 & 0.03 & 0.04 & 0.01 \\
\bottomrule\noalign{}
\end{longtable}
\end{column}
\end{columns}
\end{frame}

\begin{frame}{Independence of rows and columns}
\protect\hypertarget{independence-of-rows-and-columns}{}
\begin{itemize}
\tightlist
\item
  Let \(\symbf{r}\) be the vector of row totals,
  i.e.~\(r_i=\sum_{j=1}^J P_{ij} = \symbf{P} \symbf{1}\)
\item
  Let \(\symbf{c}\) be the vector of column totals,
  i.e.~\(c_j=\sum_{i=1}^I P_{ij} = \symbf{P}' \symbf{1}\)
\item
  Then if the rows are independent of the cells, we have that
\end{itemize}

\begin{align*}
p_{ij} &= r_ic_j,
\implies \symbf{P}_{\mathrm{ind}} = \symbf{r}\symbf{c}'
\end{align*}

\begin{longtable}[]{@{}lrrrr@{}}
\toprule\noalign{}
& none & light & medium & heavy \\
\midrule\noalign{}
\endhead
SM & 0.02 & 0.01 & 0.02 & 0.01 \\
JM & 0.03 & 0.02 & 0.03 & 0.01 \\
SE & 0.08 & 0.06 & 0.08 & 0.03 \\
JE & 0.14 & 0.11 & 0.15 & 0.06 \\
SC & 0.04 & 0.03 & 0.04 & 0.02 \\
\bottomrule\noalign{}
\end{longtable}
\end{frame}

\begin{frame}{Matrix of residuals}
\protect\hypertarget{matrix-of-residuals}{}
\begin{itemize}
\tightlist
\item
  Under the assumption of independence, we can calculate residuals:
\end{itemize}

\[
\symbf{P} - \symbf{P}_{\mathrm{ind}} = \symbf{P} - \symbf{r}\symbf{c}'.
\]

\begin{itemize}
\tightlist
\item
  Continuing the smoking example, we then have
\end{itemize}

\begin{columns}[T]
\begin{column}{0.45\textwidth}
\begin{center} 

$\symbf{P}$

\end{center}

\vspace{0.05cm}

\begin{longtable}[]{@{}lrrrr@{}}
\toprule\noalign{}
& none & light & medium & heavy \\
\midrule\noalign{}
\endhead
SM & 0.02 & 0.01 & 0.02 & 0.01 \\
JM & 0.02 & 0.02 & 0.04 & 0.02 \\
SE & 0.13 & 0.05 & 0.06 & 0.02 \\
JE & 0.09 & 0.12 & 0.17 & 0.07 \\
SC & 0.05 & 0.03 & 0.04 & 0.01 \\
\bottomrule\noalign{}
\end{longtable}
\end{column}

\begin{column}{0.025\textwidth}
\end{column}

\begin{column}{0.525\textwidth}
\begin{center}

$\symbf{P} - \symbf{r}\symbf{c}'$

\end{center}

\vspace{0.05cm}

\begin{longtable}[]{@{}lrrrr@{}}
\toprule\noalign{}
& none & light & medium & heavy \\
\midrule\noalign{}
\endhead
SM & 0.003 & -0.003 & -0.003 & 0.003 \\
JM & -0.009 & -0.006 & 0.006 & 0.009 \\
SE & 0.046 & -0.010 & -0.023 & -0.014 \\
JE & -0.051 & 0.018 & 0.025 & 0.008 \\
SC & 0.011 & 0.001 & -0.005 & -0.006 \\
\bottomrule\noalign{}
\end{longtable}
\end{column}
\end{columns}

\begin{itemize}
\tightlist
\item
  Residuals are naturally larger for the more abundant rows (employee
  ranks)
\end{itemize}
\end{frame}

\begin{frame}{Standardised residuals}
\protect\hypertarget{standardised-residuals}{}
\begin{itemize}
\tightlist
\item
  To avoid the more abundant rows and columns from dominating downstream
  analyses, we normalise by row and column size.
\item
  For each residual \(P_{ij} - P_{\mathrm{ind}}_{ij}\), we standardise
  by
\end{itemize}

\[
\frac{P_{ij} - P_{\mathrm{ind}}_{ij}}{\sqrt{r}_i\sqrt{c_j}}
\]

\begin{itemize}
\tightlist
\item
  Define the diagonal matrices \(\symbf{D}_r=\mathrm{diag}(\symbf{r})\)
  and \(\symbf{D}_c=\mathrm{diag}(\symbf{c})\).
\item
  We then have that the matrix of standardised residuals is given by
\end{itemize}

\[
\symbf{S} = \symbf{D}_r^{-1/2}(\symbf{P} - \symbf{P}_{\mathrm{ind}} ) \symbf{D}_c^{-1/2}.
\]
\end{frame}



\end{document}
